% Options for packages loaded elsewhere
\PassOptionsToPackage{unicode}{hyperref}
\PassOptionsToPackage{hyphens}{url}
\PassOptionsToPackage{dvipsnames,svgnames,x11names}{xcolor}
%
\documentclass[
  11pt,
]{article}

\usepackage{amsmath,amssymb}
\usepackage{iftex}
\ifPDFTeX
  \usepackage[T1]{fontenc}
  \usepackage[utf8]{inputenc}
  \usepackage{textcomp} % provide euro and other symbols
\else % if luatex or xetex
  \usepackage{unicode-math}
  \defaultfontfeatures{Scale=MatchLowercase}
  \defaultfontfeatures[\rmfamily]{Ligatures=TeX,Scale=1}
\fi
\usepackage{lmodern}
\ifPDFTeX\else  
    % xetex/luatex font selection
\fi
% Use upquote if available, for straight quotes in verbatim environments
\IfFileExists{upquote.sty}{\usepackage{upquote}}{}
\IfFileExists{microtype.sty}{% use microtype if available
  \usepackage[]{microtype}
  \UseMicrotypeSet[protrusion]{basicmath} % disable protrusion for tt fonts
}{}
\makeatletter
\@ifundefined{KOMAClassName}{% if non-KOMA class
  \IfFileExists{parskip.sty}{%
    \usepackage{parskip}
  }{% else
    \setlength{\parindent}{0pt}
    \setlength{\parskip}{6pt plus 2pt minus 1pt}}
}{% if KOMA class
  \KOMAoptions{parskip=half}}
\makeatother
\usepackage{xcolor}
\usepackage[margin=1in]{geometry}
\setlength{\emergencystretch}{3em} % prevent overfull lines
\setcounter{secnumdepth}{5}
% Make \paragraph and \subparagraph free-standing
\ifx\paragraph\undefined\else
  \let\oldparagraph\paragraph
  \renewcommand{\paragraph}[1]{\oldparagraph{#1}\mbox{}}
\fi
\ifx\subparagraph\undefined\else
  \let\oldsubparagraph\subparagraph
  \renewcommand{\subparagraph}[1]{\oldsubparagraph{#1}\mbox{}}
\fi


\providecommand{\tightlist}{%
  \setlength{\itemsep}{0pt}\setlength{\parskip}{0pt}}\usepackage{longtable,booktabs,array}
\usepackage{calc} % for calculating minipage widths
% Correct order of tables after \paragraph or \subparagraph
\usepackage{etoolbox}
\makeatletter
\patchcmd\longtable{\par}{\if@noskipsec\mbox{}\fi\par}{}{}
\makeatother
% Allow footnotes in longtable head/foot
\IfFileExists{footnotehyper.sty}{\usepackage{footnotehyper}}{\usepackage{footnote}}
\makesavenoteenv{longtable}
\usepackage{graphicx}
\makeatletter
\def\maxwidth{\ifdim\Gin@nat@width>\linewidth\linewidth\else\Gin@nat@width\fi}
\def\maxheight{\ifdim\Gin@nat@height>\textheight\textheight\else\Gin@nat@height\fi}
\makeatother
% Scale images if necessary, so that they will not overflow the page
% margins by default, and it is still possible to overwrite the defaults
% using explicit options in \includegraphics[width, height, ...]{}
\setkeys{Gin}{width=\maxwidth,height=\maxheight,keepaspectratio}
% Set default figure placement to htbp
\makeatletter
\def\fps@figure{htbp}
\makeatother
% definitions for citeproc citations
\NewDocumentCommand\citeproctext{}{}
\NewDocumentCommand\citeproc{mm}{%
  \begingroup\def\citeproctext{#2}\cite{#1}\endgroup}
\makeatletter
 % allow citations to break across lines
 \let\@cite@ofmt\@firstofone
 % avoid brackets around text for \cite:
 \def\@biblabel#1{}
 \def\@cite#1#2{{#1\if@tempswa , #2\fi}}
\makeatother
\newlength{\cslhangindent}
\setlength{\cslhangindent}{1.5em}
\newlength{\csllabelwidth}
\setlength{\csllabelwidth}{3em}
\newenvironment{CSLReferences}[2] % #1 hanging-indent, #2 entry-spacing
 {\begin{list}{}{%
  \setlength{\itemindent}{0pt}
  \setlength{\leftmargin}{0pt}
  \setlength{\parsep}{0pt}
  % turn on hanging indent if param 1 is 1
  \ifodd #1
   \setlength{\leftmargin}{\cslhangindent}
   \setlength{\itemindent}{-1\cslhangindent}
  \fi
  % set entry spacing
  \setlength{\itemsep}{#2\baselineskip}}}
 {\end{list}}
\usepackage{calc}
\newcommand{\CSLBlock}[1]{\hfill\break\parbox[t]{\linewidth}{\strut\ignorespaces#1\strut}}
\newcommand{\CSLLeftMargin}[1]{\parbox[t]{\csllabelwidth}{\strut#1\strut}}
\newcommand{\CSLRightInline}[1]{\parbox[t]{\linewidth - \csllabelwidth}{\strut#1\strut}}
\newcommand{\CSLIndent}[1]{\hspace{\cslhangindent}#1}

\makeatletter
\@ifpackageloaded{caption}{}{\usepackage{caption}}
\AtBeginDocument{%
\ifdefined\contentsname
  \renewcommand*\contentsname{Table of contents}
\else
  \newcommand\contentsname{Table of contents}
\fi
\ifdefined\listfigurename
  \renewcommand*\listfigurename{List of Figures}
\else
  \newcommand\listfigurename{List of Figures}
\fi
\ifdefined\listtablename
  \renewcommand*\listtablename{List of Tables}
\else
  \newcommand\listtablename{List of Tables}
\fi
\ifdefined\figurename
  \renewcommand*\figurename{Figure}
\else
  \newcommand\figurename{Figure}
\fi
\ifdefined\tablename
  \renewcommand*\tablename{Table}
\else
  \newcommand\tablename{Table}
\fi
}
\@ifpackageloaded{float}{}{\usepackage{float}}
\floatstyle{ruled}
\@ifundefined{c@chapter}{\newfloat{codelisting}{h}{lop}}{\newfloat{codelisting}{h}{lop}[chapter]}
\floatname{codelisting}{Listing}
\newcommand*\listoflistings{\listof{codelisting}{List of Listings}}
\makeatother
\makeatletter
\makeatother
\makeatletter
\@ifpackageloaded{caption}{}{\usepackage{caption}}
\@ifpackageloaded{subcaption}{}{\usepackage{subcaption}}
\makeatother
\ifLuaTeX
  \usepackage{selnolig}  % disable illegal ligatures
\fi
\usepackage{bookmark}

\IfFileExists{xurl.sty}{\usepackage{xurl}}{} % add URL line breaks if available
\urlstyle{same} % disable monospaced font for URLs
\hypersetup{
  pdftitle={Final Project Report},
  pdfauthor={Solomon Nyamekye (SN66)},
  colorlinks=true,
  linkcolor={blue},
  filecolor={Maroon},
  citecolor={Blue},
  urlcolor={Blue},
  pdfcreator={LaTeX via pandoc}}

\title{Final Project Report}
\author{Solomon Nyamekye (SN66)}
\date{Tue., Apr.~30}

\begin{document}
\maketitle

\section{Introduction}\label{introduction}

The project seek to investigate the impact of uncertainties from Sea
level rise, storm surge, discount rates, and insurance premiums on our
annual expected damages in the house elevation problem.

Premium = E{[}losses{]} + V.stdev {[}loses{]}

\subsection{Problem Statement}\label{problem-statement}

Adding insurance premium in the house elevation problem to explore it's
impact on expected annual damages.

Using probabilitic approach to explore how these uncertainties affects
the expected annual damage Considering insrance premium in our model
help us explore the trade-offs between upfront cost or borrowing and
insurance costs needs good premiums

Clearly define the problem statement that your chosen feature aims to
address. Explain the significance of this problem in the context of
climate risk management.

\subsection{Selected Feature}\label{selected-feature}

\begin{enumerate}
\def\labelenumi{\arabic{enumi}.}
\tightlist
\item
  Policy search; to explore how various input variables impact expected
  losses
\item
  Optimization; to find Optimal elevation level
\item
  Sequencial decision making to decide when to elevate after setting a
  threshold for insurance premium.
\end{enumerate}

Describe the feature you have selected to add to the existing
decision-support tool. Discuss how this feature relates to the problem
statement and its potential to improve climate risk assessment.

\section{Literature Review}\label{literature-review}

Provide a brief overview of the theoretical background related to your
chosen feature. Cite at least two relevant journal articles to support
your approach (see
\href{https://quarto.org/docs/authoring/footnotes-and-citations.html}{Quarto
docs} for help with citations). Explain how these articles contribute to
the justification of your selected feature.

\section{Methodology}\label{methodology}

\subsection{Implementation}\label{implementation}

You should make your modifications in either the \texttt{HouseElevation}
or \texttt{ParkingGarage} module. Detail the steps taken to implement
the selected feature and integrate it into the decision-support tool.
Include code snippets and explanations where necessary to clarify the
implementation process.

\subsection{Validation}\label{validation}

As we have seen in labs, mistakes are inevitable and can lead to
misleading results. To minimize the risk of errors making their way into
final results, it is essential to validate the implemented feature.
Describe the validation techniques used to ensure the accuracy and
reliability of your implemented feature. Discuss any challenges faced
during the validation process and how they were addressed.

\section{Results}\label{results}

Present the results obtained from the enhanced decision-support tool.
Use tables, figures, and visualizations to clearly communicate the
outcomes. Provide sufficient detail to demonstrate how the implemented
feature addresses the problem statement. Use the
\texttt{\#\textbar{}\ output:\ false} and/or
\texttt{\#\textbar{}\ echo:\ false} tags to hide code output and code
cells in the final report except where showing the output (e.g.g, a
plot) or the code (e.g., how you are sampling SOWs) adds value to the
discussion. You may have multiple subsections of results, which you can
create using \texttt{\#\#}.

\section{Conclusions}\label{conclusions}

\subsection{Discussion}\label{discussion}

Analyze the implications of your results for climate risk management.
Consider the context of the class themes and discuss how your findings
contribute to the understanding of climate risk assessment. Identify any
limitations of your approach and suggest potential improvements for
future work.

\subsection{Conclusions}\label{conclusions-1}

Summarize the key findings of your project and reiterate the
significance of your implemented feature in addressing the problem
statement. Discuss the broader implications of your work for climate
risk management and the potential for further research in this area.

\section{References}\label{references}

\phantomsection\label{refs}
\begin{CSLReferences}{0}{1}
\end{CSLReferences}



\end{document}
